% !TeX root = main.tex
\documentclass[a4paper,alpha-refs]{scr-contemporary}

\usepackage{xcolor}
\usepackage{graphicx}
\usepackage{siunitx}
\usepackage{amsmath}
\usepackage{graphicx}
\usepackage{caption}
\usepackage{newpxmath}
\usepackage{ragged2e}
\usepackage{booktabs}
\usepackage{tikz}

\captionsetup{justification=centering}
\let\mathscr\undefined

% Surrey Capital Research Specifics
\jname{Surrey Capital Research} % Set the journal name to Surrey Capital Research

\secondlogo{../images/scr-logo.pdf} % Surrey Capital Research logo
\jlogo{../images/surrey-logo.pdf} % University of Surrey logo


\title{A Deep Dive into Portfolio Optimisation}

\author{\fontsize{12pt}{12pt}\selectfont Riccardo di Silvio, Jack Humphries, Sofiya Kolokolnikova, Freya Cullen, Alex Inskip}

\affil{Surrey Capital Research, University of Surrey}
\papercat{Quantitative Research}

\runningauthor{Surrey Capital Research}

\jvolume{01}
\jnumber{01}
\jyear{2026}

\numberwithin{equation}{section}

\begin{document}

\begin{frontmatter}
\maketitle

\begin{abstract}

\textbf{Background}:
  Selecting an optimal portfolio allocation across a universe of assets is a central problem in 
  investment management. A wide range of allocation strategies have been proposed, from naive              
  diversification to theoretically grounded optimisation frameworks, yet their relative out-of-sample      
  performance remains contested, particularly across varying market regimes.  
  This study empirically compares four portfolio strategies to assess whether model sophistication translates 
  into superior investment performance. \\
\vspace{0.5mm}
\textbf{Methodology}:
  This study implements and compares four portfolio strategies: equal weight, Markowitz mean-variance, 
  Black-Litterman and Risk Parity applied to an 18-asset UK multi-asset portfolio comprising 15 FTSE 100 equities, 
  a UK government bond ETF, a gold ETC and a broad commodities ETF, over the period January 2015 to December 2025. 
  Strategies are evaluated using a monthly rebalancing framework and assessed across annualised return, volatility, 
  Sharpe ratio, maximum drawdown and portfolio turnover. \\
\vspace{0.5mm}
  \textbf{Results}:
  We find results broadly consistent with DeMiguel et al. (2009), with the naive benchmark 
  proving difficult to displace on a risk-adjusted basis over the full sample period. The optimisation 
  strategies exhibit meaningfully different risk profiles, however, suggesting that Sharpe ratio alone does not 
  fully capture the practical trade-offs between approaches. \\
\end{abstract}

\end{frontmatter}

\begin{keypoints*}
\begin{itemize}
  \item The equal-weight portfolio is robust because it does not rely on 
  estimated parameters and is therefore unaffected by estimation error.
  \item The Efficient Frontier is the set of portfolios that deliver the minimum 
  possible variance for each level of expected return.
  \item Empirical evidence suggests that many optimisation models fail to 
  consistently outperform the equal-weight portfolio out of sample.
\end{itemize}
\end{keypoints*}

\section{Introduction}
\subsection{The Portfolio Allocation Problem}
\indent \indent The portfolio allocation problem is the fundamental question in investment management. 
It asks: How should capital be distributed across available investment opportunities to best achieve 
an investor's objectives? \\
\indent The problem was intractable until 1952, when Harry Markowitz published \textit{Portfolio Selection} in the 
\textit{Journal of Finance} and turned it into a clean optimisation problem, and the “efficient frontier” 
gave investors a principled answer. Markowitz's mean variance model minimises portfolio variance for a given 
level of expected return. It was revolutionary in the consideration of variance as the risk, standing as a 
hallmark of modern portfolio theory. In practice however, the model has proven difficult to implement reliably. \\
\indent Michaud (1989) challenged this view directly, arguing that small estimation errors in expected 
returns produce wildly unstable portfolio recommendations. \\
\indent In 2009, DeMiguel et al. delivered a provocative result: he found that even sophisticated strategies 
incorporating shrinkage estimators and Bayesian methods failed to consistently outperform 1/N out-of-sample, 
suggesting that estimation error dominates theoretical optimality for typical portfolio problems. This tension 
between the theoretical elegance of optimisation and its empirical fragility motivates the present study. 


\subsection{Research Objectives}
\indent \indent This study addresses three main questions:
\begin{enumerate}
    \item Do optimisation models outperform a naive 1/N benchmark
    out-of-sample, applied to a multi-asset UK portfolio over 
    a ten-year period? 
    \item Which model offers the best risk-adjusted performance, 
    as measured by the Sharpe ratio?
    \item How does model performance vary across distinct market
    regimes: the Brexit referendum (2016), the COVID-19 crash (2020),
    and the 2022 rate-hiking cycle?
\end{enumerate}

\subsection{Structure of the Report}
\indent \indent The remainder of this report is organised as follows: \newline
Section 2 reviews the theoretical foundations of each model 
and derives their key mathematical results. Section 3 
describes the data, implementation choices, and backtesting 
framework. Section 4 presents the empirical results, including 
full-period performance metrics and a breakdown by market 
regime. Section 5 discusses the findings and their practical 
implications. Section 6 concludes.

\section{Literature and Theoretical Background}
\label{sec:theory}
\subsection{Notation}
\indent \indent Some common notation that is used across this paper is the following:

\begin{center}
\begin{tabular}{l l}
\toprule
\textbf{Symbol} & \textbf{Description} \\
\midrule
$N$ & Number of assets in the portfolio \\
$\mathbf{w} \in \mathbb{R}^N$ & Vector of portfolio weights \\
$\boldsymbol{\mu} \in \mathbb{R}^N$ & Vector of expected returns \\
$\boldsymbol{\Sigma} \in \mathbb{R}^{N \times N}$ & Covariance matrix of asset returns \\
$\mathbf{1} \in \mathbb{R}^N$ & Vector of ones \\
$r_f \in \mathbb{R}$ & Risk-free rate \\
\bottomrule
\end{tabular}
\end{center}

\subsection{The equal weight portfolio}

\indent \indent The equal-weight (1/N) portfolio assigns an identical weight of 1/N to each asset in the portfolio, holding N assets, regardless of any asset-specific 
characteristics such as return, volatility, or correlation. 
\citet{demiguel_optimal_2009} tested 14 optimisation models across 7 datasets. None consistently outperformed the naive equal-weight portfolio: 
\begin{equation}
  w_i = \frac{1}{N} \quad \text{for} \quad 1 \leq i \leq N
\end{equation}
\indent But why? When expected returns are estimated from historical data in order to determine optimal portfolio weights, the estimates 
improve as more data is collected, but they do so slowly. \newline
Formally, the estimation error decays as
\begin{equation}
  \hat{\mu} - \mu = \mathcal{O}\!\left(\frac{1}{\sqrt{T}}\right)
\end{equation}
where $T$ is the number of observations.

Because the error shrinks at rate $\frac{1}{\sqrt{T}}$, halving the 
error requires four times as much data, meaning that with monthly data 
and realistic return distributions, approximately 500 years of observations 
are required for statistically reliable estimates, especially since these
errors are amplified through the term $\Sigma^{-1}\hat{\mu}$. \newline
\indent The equal-weight portfolio sidesteps this problem entirely: by assigning a fixed weight to each asset, it requires no parameter estimation 
and is therefore immune to estimation error by construction. \newline
The cost of this simplicity, however, is that the portfolio ignores all information about assets' risk and return characteristics, makes 
no attempt to diversify risk efficiently and cannot adapt to changing market conditions.


% ─────────────────────────────────────────────
\subsection{Markowitz's mean-variance optimisation}
\label{sec:mvo}

\indent \indent \citet{markowitz_portfolio_1952} formulated \textit{Portfolio Selection} in 1952, approaching the asset 
allocation problem as a constrained optimisation problem: minimise 
portfolio variance for a given level of expected return, the foundation of the mean-variance model.
His work stands as a hallmark of modern portfolio theory.\newline
\indent Formally, the problem states:
\begin{equation}                                         
  \begin{aligned}
  \min_{\mathbf{w}} \hat{\sigma} =
  \mathbf{w}^\top \hat{\boldsymbol{\Sigma}}\ \mathbf{w}
  \quad \text{s.t.} \quad
  \mathbf{w}^\top \boldsymbol{\mu} = \mu_p
  \quad \text{;} \quad 
  \mathbf{w}^\top\mathbf{1} = 1
  \end{aligned}
\end{equation}

with $\mu_p$ as the scalar target portfolio return.

The first constraint, also known as the return constraint, fixes the portfolio to a 
specific point at the target level $\mu_p$, pinning the solution to a specific point on the "Efficient Frontier".
By varying $\mu_p$ across all feasible values, the complete set of minimum-variance 
portfolios - the Efficient Frontier - is traced out.
The second constraint, i.e. the budget constraint, ensures the weights represent a valid
allocation of wealth, hence that the sum of all weights is always equal to $1$. \newline
\indent It is important to note that no domain-restriction condition is imposed on the budget constraint,
each $w_i$ may take any real value, in particular, negative ones.
Practically speaking, a negative weight corresponds to a short position in the associated asset.

\begin{figure}[htbp]
    \centering
    \includegraphics[width=0.45\textwidth]{../images/plots/theory/efficient_frontier.pdf}
    \caption{Efficient Frontier estimated using sample covariance}
    \label{fig:efficient_frontier}
\end{figure}

Because the objective function is quadratic and the constraints are linear equalities, we cannot solve the problem by direct substitution 
(two constraints, n unknowns). We therefore use the method of Lagrange multipliers. 
The Lagrangian formulation is:

\begin{equation}
  \mathcal{L}(\mathbf{w}, \lambda, \gamma)
      = \mathbf{w}^\top \boldsymbol{\Sigma} \mathbf{w}
      - \lambda\!\left(\mathbf{w}^\top \boldsymbol{\mu} - \mu_p\right)
      - \gamma\!\left(\mathbf{1}^\top \mathbf{w} - 1\right)
\end{equation}

where $\lambda$ and $\gamma$ are the Lagrange multipliers associated with the return and budget constraints, respectively. 
Intuitively, $\lambda$ captures the marginal cost of requiring a higher return (i.e. how much additional variance must be accepted 
per unit increase in $\mu_p$), and $\gamma$ enforces full investment.

Taking the first-order condition:
\begin{equation}
  \frac{\partial \mathcal{L}}{\partial \mathbf{w}} = 2\boldsymbol{\Sigma} \mathbf{w} - \lambda \boldsymbol{\mu} - \gamma \mathbf{1} = \mathbf{0}
\end{equation}

Solving gives:
\begin{equation}
  \mathbf{w}_p = \frac{\lambda}{2}\,\boldsymbol{\Sigma}^{-1}\boldsymbol{\mu} + \frac{\gamma}{2}\,\boldsymbol{\Sigma}^{-1}\mathbf{1}
\end{equation}

Imposing the budget and return constraints leads to the following two-fund theorem: all efficient portfolios can be expressed as a 
linear combination of any two distinct efficient portfolios.
\begin{equation}
  \mathbf{w}_A = \frac{\boldsymbol{\Sigma}^{-1}\mathbf{1}}{\mathbf{1}^\top \boldsymbol{\Sigma}^{-1}\mathbf{1}}
  \qquad
  \mathbf{w}_B = \frac{\boldsymbol{\Sigma}^{-1}\boldsymbol{\mu}}{\mathbf{1}^\top \boldsymbol{\Sigma}^{-1}\boldsymbol{\mu}}
\end{equation}

Note that $\mathbf{w}_A$ is the \emph{global minimum variance} (GMV) portfolio, 
the leftmost point on the efficient frontier, obtained by dropping the 
return constraint and minimising variance subject only to full investment:
\begin{equation}
  \mathbf{w}_{\mathrm{GMV}}
    = \frac{\boldsymbol{\Sigma}^{-1}\mathbf{1}}{\mathbf{1}^\top \boldsymbol{\Sigma}^{-1}\mathbf{1}}
\end{equation}

And the \emph{maximum Sharpe ratio} portfolio is:
\begin{equation}
  \mathbf{w}_{\mathrm{MSR}}
    = \frac{\boldsymbol{\Sigma}^{-1}(\boldsymbol{\mu} - r_f \mathbf{1})}
           {\mathbf{1}^\top \boldsymbol{\Sigma}^{-1}(\boldsymbol{\mu} - r_f \mathbf{1})}
  \label{eq:msr}
\end{equation}
\indent Despite its theoretical elegance, Markowitz optimisation suffers from a well-documented problem. 
As mentioned above, the solution depends critically on $\boldsymbol{\mu}$ and $\boldsymbol{\Sigma}$, both estimated from historical data. \newline
\indent Small errors in $\mathbf{\mu}$ can cause large shifts in optimal weights.
\citet{michaud_markowitz_1989} characterised the optimiser as an "error maximiser": it systematically overweights assets with 
overestimated returns and underestimated variances.

\begin{keypoints*}
\begin{itemize}
  \item Black-Litterman uses market-cap weights as a neutral 
  starting point for expected returns.
  \item Subjective investor views are blended with the equilibrium
  prior through a Bayesian update.
  \item The optimisation step in Black-Litterman is the same as MVO
  with $\boldsymbol{\mu}_{BL}$ replacing the sample mean.                                                    
\end{itemize}
\end{keypoints*}

% ─────────────────────────────────────────────
\subsection{Black-Litterman and its Bayesian approach}

\indent \indent The Black-Litterman (BL) model was developed by Fischer Black and 
Robert Litterman at Goldman Sachs in the early 1990s 
to address the instability of Mean-Variance Optimisation \citep{black_asset_1991, black_global_1992}.

\indent The BL model addresses two shortcomings of classic mean-variance optimisation:
\begin{enumerate}
  \vspace{-2mm}
  \item What expected returns should be used? \newline
  Rather than relying on historical data, BL uses returns implied by current 
  market-cap weights as the neutral starting point.
  \item How should the views be incorporated? \newline
  BL offers a formal Bayesian framework for blending subjective investor views 
  with this equilibrium prior, weighting each source inversely by its uncertainty.
  \vspace{-2mm}
\end{enumerate}

\noindent \emph{Step 1. Equilibrium Returns} \\
\indent The model takes as its starting point the equilibrium expected returns implied 
by the market portfolio, i.e. the portfolio where each asset is held in proportion
to its market capitalisation.
Assuming the market portfolio is mean-variance efficient, the equilibrium excess returns are:
\begin{equation}
\Pi = \delta \Sigma w_{m}
\end{equation}
where $w_m$ is the market-cap weight vector, $\Sigma$ is the covariance 
matrix of asset returns and $\delta$ is the risk-aversion coefficient
\begin{equation}
  \delta = \frac{r_m - r_f}{\sigma_m^2}
\end{equation}
with $r_m$ the return of the market portfolio and $\sigma^2$ its variance. \\

\noindent \emph{Step 2. Express Views} \\
\indent A view can be absolute or relative, for example "asset A will return 5\%" or 
"asset A will outperform asset B by 2\%".
Each view has two components: the expected return itself and a confidence 
level attached to it. \newline
\indent Formally, k views are expressed through the following system:
\begin{equation}
  P \boldsymbol{\mu} = \boldsymbol{Q} + \boldsymbol{\varepsilon}
  \qquad \boldsymbol{\varepsilon} \sim \mathcal{N}(\mathbf{0}, \Omega)
\end{equation}
where:
\begin{center}
\begin{tabular}{l l}
\toprule
\textbf{Symbol} & \textbf{Description} \\
\midrule
$P \in \mathbb{R}^{k \times n}$ & \emph{Pick matrix}: encodes which assets views involve \\
$\boldsymbol{\mu} \in \mathbb{R}^k$ & Vector of expected returns \\
$\boldsymbol{Q} \in \mathbb{R}^k$ & Expected view returns \\
$\Omega \in \mathbb{R}^{k \times k}$ & View uncertainty matrix (diagonal) \\
$\boldsymbol{\varepsilon} \sim \mathcal{N}(\mathbf{0}, \Omega)$ & View error: deviation of true returns from stated views \\
\bottomrule
\end{tabular}
\end{center}

\indent Assets are ranked by their return over the previous 12 months, excluding the most 
recent month to eliminate short-term reversal effects. The highest-return tercile forms
the winner portfolio, while the lowest-return decile constitutes the loser portfolio. \\
\indent The view is expressed as:
\begin{equation}
    \boldsymbol{p}_{k}^{\top} \boldsymbol{\mu} = q_k
\end{equation}
\indent $\boldsymbol{p}_k$ assigns equal positive weights to winner assets and equal negative weights 
to loser assets, forming a long-short momentum view. The expected return 
spread $q_k$ is estimated over a rolling window at each rebalancing date as asset rankings change. \\

\noindent \emph{Step 3. Bayesian Update} \\
\indent The prior on expected returns is:
\begin{equation}
    \boldsymbol{\mu} \sim \mathcal{N}(\boldsymbol{\Pi}, \tau \boldsymbol{\Sigma})
\end{equation}
\indent This encodes the belief that expected returns are centred on the equilibrium 
vector $\boldsymbol{\Pi}$, with uncertainty scaled by $\tau$. Combining this prior 
with the view system via Bayes' theorem yields the posterior distribution over 
expected returns:
\begin{equation}
    \boldsymbol{\mu} \mid \boldsymbol{Q} \sim \mathcal{N}(\boldsymbol{\mu_{BL}}, \boldsymbol{\Sigma_{BL}})
\end{equation}
where the posterior mean is:
\begin{equation}
    \boldsymbol{\mu_{BL}} = 
    \left[(\tau \Sigma)^{-1} 
    + P^{\top}\Omega^{-1}P\right]^{-1}\left[(\tau \Sigma)^{-1}\boldsymbol{\Pi} 
    + P^{\top}\Omega^{-1}\boldsymbol{Q}\right]
\end{equation}
\indent The posterior mean $\boldsymbol{\mu}_{BL}$ is a precision-weighted average of the 
equilibrium prior $\boldsymbol{\Pi}$ and the investor's views $\boldsymbol{Q}$, with each 
source weighted inversely by its uncertainty. 
When views are diffuse (large $\boldsymbol{\Omega}$), $\boldsymbol{\mu}_{BL}$ 
remains close to $\boldsymbol{\Pi}$. 
If the views are held with high confidence (small $\boldsymbol{\Omega}$), 
$\boldsymbol{\mu}_{BL}$ tilts toward $\mathbf{Q}$. \newline

This blending property is central to the appeal of the BL framework: 
Rather than overriding the equilibrium prior, views are incorporated gradually 
and in proportion to their confidence. This acts as a natural regulariser, because the 
prior pulls weights toward the market portfolio, the model is far less prone to the 
concentrated, unstable allocations that plague classical MVO. \\

\noindent \emph{Step 4. Portfolio Optimisation} \\
\indent The optimal portfolio is obtained by applying the same Sharpe-maximising 
objective derived in Section~\ref{sec:mvo} (equation~\ref{eq:msr}), 
substituting $\boldsymbol{\mu}_{\text{BL}}$ in place of the sample mean $\boldsymbol{\mu}$.


\subsection{Risk Parity}

\begin{keypoints*}
\begin{itemize}
  \item Risk Parity allocates capital such that every asset contributes                                       
    equally to total portfolio volatility, not equally to capital.
  \item Unlike MVO and Black-Litterman, Risk Parity requires no estimate
    of expected returns: allocations depend only on the covariance structure.
  \item The equal risk contribution condition $w_i(\boldsymbol{\Sigma}\mathbf{w})_i = c$
    for all $i$ has no closed-form solution and in general and is solved numerically.
\end{itemize}
\end{keypoints*}

\indent \indent The modern Risk Parity framework was popularised by Ray Dalio 
at Bridgewater Associates and underpins the All Weather strategy introduced 
in the 1990s. The approach was subsequently
formalised by \citet{maillard_properties_2010}, who derived the mathematical
conditions for equal risk contribution portfolios. \\
\indent The central motivation
departs entirely from the MVO and BL frameworks: rather than optimising
expected return for a given level of risk, Risk Parity ignores return estimates
altogether and instead focuses on how risk is distributed across the portfolio.

\indent Risk Parity addresses a structural limitation shared by both MVO and
the equal-weight portfolio:
\begin{enumerate}
  \vspace{-2mm}
  \item In MVO and BL, allocations depend on estimated expected returns, which
  are noisy and prone to estimation error.
  \item In equal-weight portfolios, capital is diversified equally but risk is not.
  Assets with high volatility dominate total portfolio risk despite receiving the
  same capital allocation.
  \vspace{-2mm}
\end{enumerate}

\noindent \emph{Risk Decomposition} \\
\indent Portfolio volatility $\sigma_p = \sqrt{\mathbf{w}^\top \boldsymbol{\Sigma} \mathbf{w}}$
is a homogeneous function of degree 1 in $\mathbf{w}$. By Euler's homogeneous
function theorem, it admits the following decomposition:
\begin{equation}
  \sigma_p = \sum_{i=1}^{N} w_i \frac{\partial \sigma_p}{\partial w_i}
\end{equation}
The \emph{marginal risk contribution} (MRC) of asset $i$ is:
\begin{equation}
  \text{MRC}_i = \frac{\partial \sigma_p}{\partial w_i}
  = \frac{(\boldsymbol{\Sigma}\mathbf{w})_i}{\sigma_p}
\end{equation}
and the total \emph{risk contribution} (RC) of asset $i$ is:
\begin{equation}
  \text{RC}_i = w_i \cdot \text{MRC}_i
  = \frac{w_i\,(\boldsymbol{\Sigma}\mathbf{w})_i}{\sigma_p}
\end{equation}
with the useful property that $\sum_{i=1}^{N} \text{RC}_i = \sigma_p$. \\

\noindent \emph{The Risk Parity Condition} \\
\indent The portfolio is said to be \emph{risk-balanced} when every asset
contributes equally to total portfolio volatility:
\begin{equation}
  \text{RC}_i = \frac{\sigma_p}{N} \quad \forall\, i
\end{equation}
Since $\sigma_p$ appears on both sides, this reduces to the equivalent condition:
\begin{equation}
  w_i\,(\boldsymbol{\Sigma}\mathbf{w})_i = c \quad \forall\, i
  \label{eq:rp}
\end{equation}
for some constant $c > 0$, subject to $\mathbf{1}^\top \mathbf{w} = 1$ and $\mathbf{w} \geq \mathbf{0}$.
This system has no closed-form solution in general and is solved numerically;
the implementation details are described in Section~\ref{sec:methodology}. \\

\begin{figure}[htbp]
    \centering
    \includegraphics[width=0.45\textwidth]{../images/plots/theory/risk_parity_illustration.pdf}
    \caption{Risk parity weights as a function of asset variance}
    \label{fig:risk_parity_illustration}
\end{figure}

\noindent \emph{Special Case: Uncorrelated Assets} \\
\indent When $\boldsymbol{\Sigma}$ is diagonal, the risk parity condition reduces to
an analytic solution. Equal risk contribution then requires each asset's weight to be
proportional to the inverse of its volatility:
\begin{equation}
  w_i^{\mathrm{RP}} = \frac{1/\sigma_i}{\displaystyle\sum_{j=1}^{N} 1/\sigma_j}
\end{equation}
\indent This inverse-volatility weighting is the closed-form limit of risk parity, and also
serves as the natural initialisation point for the numerical solver in the general case.

\indent Despite its appeal, Risk Parity has a well-known structural consequence:
it mechanically overweights low-volatility assets. In a portfolio containing
both equities and government bonds, bonds receive substantially larger
capital allocations precisely because their volatility is lower, which raises
the question of whether the resulting portfolio is truly diversified in an
economic sense or merely in a statistical one.

\section{Methodology}
\label{sec:methodology}

\subsection{Data}
\label{sec:data}

\indent \indent The dataset comprises daily adjusted closing prices for 18 assets
over an 11-year period from 1 January 2015 to 31 December 2025, sourced
programmatically from \citeauthor{yahoo_finance}. The universe is constructed to reflect a
representative UK-centric multi-asset portfolio: 15 large-cap FTSE 100 equities
spanning eight sectors, supplemented by three exchange-traded products providing
exposure to UK government bonds, gold, and broad commodities. The full asset
universe is listed in Table~\ref{tab:assets}.

\begin{table}[h]
\centering
\begin{tabular}{l l l}
\toprule
\textbf{Asset Class} & \textbf{Name} & \textbf{Ticker} \\
\midrule
Banking         & HSBC                         & HSBA.L \\
                & Lloyds Banking Group         & LLOY.L \\
                & Barclays                     & BARC.L \\
Oil \& Gas      & Shell                        & SHEL.L \\
                & BP                           & BP.L   \\
Consumer Goods  & Unilever                     & ULVR.L \\
                & Tesco                        & TSCO.L \\
                & Diageo                       & DGE.L  \\
Pharmaceuticals & AstraZeneca                  & AZN.L  \\
                & GSK                          & GSK.L  \\
Mining          & Rio Tinto                    & RIO.L  \\
                & Glencore                     & GLEN.L \\
Utilities       & National Grid                & NG.L   \\
Telecoms        & Vodafone                     & VOD.L  \\
Industrials     & Rolls-Royce                  & RR.L   \\
\midrule
UK Govt Bonds   & iShares Core UK Gilts ETF    & IGLT.L \\
Precious Metals & Invesco Physical Gold ETC    & SGLD.L \\
Commodities     & WisdomTree Broad Commodities & WCOG.L \\
\bottomrule
\end{tabular}
\caption{Asset universe (18 assets)}
\label{tab:assets}
\end{table}
\indent Adjusted closing prices are used in preference to raw prices. Yahoo
Finance's adjusted close corrects for corporate actions including stock splits,
rights issues, and dividend distributions, ensuring that computed returns reflect
the true total return experienced by an investor. Missing observations arising
from non-synchronous trading calendars are handled by forward-filling the most
recent available price prior to return calculation. \\

\indent The risk-free rate is proxied by the 3-month UK Treasury Bill rate,
sourced from the Federal Reserve Economic Data database and converted to
a daily rate \citeauthor{FRED}. The rate is applied time-variably across the sample, capturing the
material shift from near-zero rates during 2015--2021 to above 5\% during the
2023--2024 tightening cycle.

\begin{keypoints*}
\begin{itemize}
  \item The universe comprises 18 assets across diversified sectors, 
  including equities, bonds, gold and commodities.
  \item Monthly rebalancing on the final trading day of each calendar 
  month is applied consistently across all four strategies.
  \item The walk-forward design of the backtester enforces a strict information
  barrier, ensuring no look-ahead bias.
\end{itemize}
\end{keypoints*}

\subsection{Model Implementation}
\label{sec:model-implementation}
\indent \indent All models are implemented from first principles in Python using
\texttt{NumPy} and \texttt{Pandas}, following an object-oriented design in which
each strategy inherits from a common abstract base class. This ensures full
transparency and reproducibility of all optimisation steps. The complete source
code is publicly available at the project repository \citep{scr_repo}. \\
\indent All models share a rolling estimation framework: at each monthly rebalancing
date, covariance and return inputs are re-estimated using the preceding 252
trading days of data. Table~\ref{tab:implementation} summarises the key
implementation parameters for each strategy.

\begin{table}[h]
\centering
\begin{tabular}{l l l l}
\toprule
\textbf{Model} & \textbf{Window} & \textbf{Constraints} & \textbf{Solver} \\
\midrule
Equal Weight    & None     & $w_i = 1/N$ & Closed-form    \\
MVO             & 252 days & Long-only   & Active-set     \\
Black-Litterman & 252 days & Long-only   & Active-set     \\
Risk Parity     & 252 days & Long-only   & Newton-Raphson \\
\bottomrule
\end{tabular}
\caption{Implementation parameters by strategy}
\label{tab:implementation}
\end{table}

\indent Equal Weight assigns $w_i = 1/N$ at each rebalancing date 
and requires no parameter estimation. \\

\indent In the Mean-Variance model, the maximum Sharpe ratio portfolio is solved 
via an active-set method: assets with negative target weights are iteratively pruned from the
investment set and the system is re-solved until all remaining weights satisfy
the long-only constraint $w_i \geq 0$. A small ridge term $\lambda = 10^{-8}$
is added to the diagonal of $\hat{\boldsymbol{\Sigma}}$ at each step to
guarantee numerical positive definiteness. \\

\indent Risk Parity solves the equal risk contribution system
(equation~\ref{eq:rp}) via Newton-Raphson iteration, initialised with
inverse-volatility weights. Convergence is typically achieved within 10
iterations. \\

\indent The Black-Litterman model requires two calibration parameters: the
prior uncertainty scalar $\tau = 0.05$ and the risk-aversion coefficient
$\delta = 3.0$. The investor views are constructed by the momentum effect
documented by \citet{jegadeesh_returns_1993}. \newline 
\indent Assets are ranked by their return over the prior 12 months excluding the 
most recent month, and a long-short view is formed between the top and bottom terciles 
at each rebalancing date. The posterior expected returns $\boldsymbol{\mu}_{BL}$
are then passed into the same active-set MVO solver used for the standard MVO
strategy.

\subsection{Backtesting Framework}
\indent \indent The four strategies are evaluated using a walk-forward backtesting 
engine to ensure zero look-ahead bias. The engine enforces a strict information 
barrier by passing only prices up to the rebalancing date; the covariance dependent 
strategies (MVO, BL, RP) then estimate over a 252 trading day sub-window. 
This distinction is already implicit in Table~\ref{tab:implementation}
(Section \ref{sec:model-implementation}). \\

\begin{figure}[htbp]                                                                                          
\centering
\begin{tikzpicture}[node distance=1.5cm, scale=0.9, every node/.style={scale=0.9}]                            
    \tikzstyle{block} = [rectangle, draw,
        text width=3.5cm, text centered, rounded corners=4pt, minimum height=1cm,
        line width=1pt, draw=jcolour]
    \tikzstyle{arrow} = [thick,->,>=stealth,color=jcolour]

    \node [block] (data) {Past Price Data\\(up to rebalance date)};
    \node [block, below of=data] (strategy) {Strategy Computes\\Target Weights};
    \node [block, below of=strategy] (rebalance) {Rebalance Portfolio};
    \node [block, below of=rebalance] (track) {Track Portfolio Value};

    \draw [arrow] (data) -- (strategy);
    \draw [arrow] (strategy) -- (rebalance);
    \draw [arrow] (rebalance) -- (track);
    \draw [arrow] (track.east) -- ++(2,0) |-
        node[near start, right, font=\footnotesize]
    {\textcolor{black}{Next Month}} (data.east);
\end{tikzpicture}
\caption{Walk-forward backtesting loop}
\label{fig:backtest_loop}
\end{figure}

\indent A starting capital of \pounds 100{,}000 is deployed on the first available 
trading day. The portfolio is thereafter rebalanced on the final trading day of each
calendar month, with all positions revalued daily at the adjusted closing price. \\

\indent The simulation assumes perfect asset divisibility (i.e. the existence of fractional
shares) and immediate execution at the closing price on each rebalancing date. Transaction
costs are not modelled; portfolio turnover is instead reported as a standalone metric, allowing
the reader to assess the practical trading burden of each strategy independently.
Bid-ask spreads and market impact are not accounted for; given that
all 18 assets are large-cap and exchange-traded, it is reasonable to assume
sufficient liquidity to absorb the portfolio's order flow without material
execution slippage. \\

\indent Each simulation run produces a \texttt{BacktestResult} object containing the daily 
equity curve, full position history and a trade log. Performance metrics are computed 
from the equity curve at the conclusion of each run and are defined in 
Section~\ref{sec:performance-metrics}.

\subsection{Performance Metrics}
\label{sec:performance-metrics}

\indent \indent Nine metrics are used to evaluate each strategy, grouped
into three categories: return, risk, and risk-adjusted. Let
$r_t = V_t / V_{t-1} - 1$ denote the daily portfolio return,
$V_0$ and $V_T$ the initial and terminal portfolio values, $T$ the total
number of trading days and $r_f$ the mean annualised risk-free rate over
the sample. Table~\ref{tab:metrics} defines each metric formally. \\

\indent Return metrics capture the magnitude of wealth creation. 
CAGR annualises the compounded total return to allow comparison across 
strategies regardless of sub-period variation. \newline 
\indent Risk metrics characterise the severity and shape of losses: 
maximum drawdown measures the worst peak-to-trough decline an investor 
would have experienced; VaR and CVaR characterise tail losses from the 
empirical return distribution without parametric assumptions. \newline 
\indent Risk-adjusted metrics normalise excess return by some measure of 
risk: the Sharpe ratio uses total volatility, while the Sortino ratio 
penalises only downside deviations, making it more appropriate when
return distributions are asymmetric.

\begin{table}[h]
\centering
\begin{tabular}{l l l}
\toprule
\textbf{Metric} & \textbf{Formula} & \textbf{Category} \\
\midrule
Total Return
    & $V_T / V_0 - 1$
    & Return \\
CAGR
    & $(V_T / V_0)^{252/T} - 1$
    & Return \\
Volatility
    & $\hat{\sigma}(r_t) \cdot \sqrt{252}$
    & Risk \\
Max Drawdown
    & $\displaystyle\min_t \left(\frac{V_t}{\max_{s \leq t} V_s} - 1\right)$
    & Risk \\
95\% VaR
    & $\text{Percentile}(r_t,\; 5\%)$
    & Risk \\
95\% CVaR
    & $\mathbb{E}[\,r_t \mid r_t \leq \text{VaR}\,]$
    & Risk \\
Sharpe Ratio
    & $(\text{CAGR} - r_f) \;/\; \hat{\sigma}_p$
    & Risk-Adjusted \\
Sortino Ratio
    & $(\text{CAGR} - r_f) \;/\; \hat{\sigma}_{\mathrm{down}}$
    & Risk-Adjusted \\
Avg.\ Monthly Turnover
    & $\displaystyle\frac{1}{M}\sum_{m=1}^{M}\sum_{i=1}^{N}|w_{i,m} - w_{i,m-1}|$
    & Portfolio \\
\bottomrule
\end{tabular}
\caption{Performance metrics used to evaluate each strategy.}
\label{tab:metrics}
\end{table}

\indent VaR and CVaR are computed from the empirical distribution of daily
returns without a parametric assumption. $\hat{\sigma}_{\mathrm{down}}$
denotes the annualised standard deviation of negative daily returns only,
so the Sortino ratio penalises downside volatility exclusively.
Average monthly turnover measures the mean total absolute weight change
across all assets at each rebalancing date, providing a strategy-level
indicator of trading activity.

\section{Results}

The results section evaluates four portfolio optimisation strategies,  Equal Weight, Risk Parity, Mean-Variance Optimisation (MVO), and Black-Litterman, across the  2015 to 2025 period, using a starting portfolio value of £100,000. The analysis is organised across four subsections, covering full-period performance, risk-return trade-offs, regime dependence, and portfolio characteristics. 
%% ─────────────────────────────────────────────
\subsection{Full-Period Performance}

Tables 1 and 2 present the return, risk, and downside risk metrics for all four strategies over the full sample period from 2015 to 2025, with Figure 3 showing the corresponding cumulative wealth curves and Figure 4 the drawdown profiles.

\begin{table}[h]

\begin{center}
\caption{Return and Risk Metrics (2015--2025)}
\label{tab:returns_risk}
\begin{tabular}{lcccc}
\toprule
\textbf{Metric} & \textbf{Equal Weight} & \textbf{Risk Parity} & \textbf{MVO} & \textbf{Black-Litterman} \\
\midrule
Total Return  & 91.57\%  & 27.34\%  & 23.56\%  & 22.17\%  \\
CAGR          &  6.08\%  &  2.22\%  &  1.94\%  &  1.83\%  \\
Volatility    & 13.94\%  &  9.39\%  & 10.31\%  & 12.03\%  \\
Sharpe Ratio  &    0.44  &    0.24  &    0.19  &    0.15  \\
Max Drawdown  & -29.68\% & -22.60\% & -22.60\% & -27.35\% \\
\bottomrule
\end{tabular}
\end{center}
\end{table}

\vspace{-8mm}
\begin{table}[h]
\centering
\caption{Downside Risk Metrics (2015--2025)}
\label{tab:downside_risk}
\begin{tabular}{lcccc}
\toprule
\textbf{Metric} & \textbf{Equal Weight} & \textbf{Risk Parity} & \textbf{MVO} & \textbf{Black-Litterman} \\
\midrule
Sortino Ratio &  0.56   &  0.29   &  0.23   &  0.18   \\
95\% VaR      & -1.28\% & -0.86\% & -0.95\% & -1.05\% \\
95\% CVaR     & -2.09\% & -1.46\% & -1.63\% & -1.84\% \\
\bottomrule
\end{tabular}
\end{table}


\begin{figure}[h]
    \centering
    \includegraphics[width=0.45\textwidth]{../images/plots/results/fig3_cumulative_wealth.pdf}
    \caption{Cumulative wealth curves for all four strategies (2015--2025), indexed to \pounds100,000 initial capital.}
    \label{fig:cumulative_wealth}
\end{figure}

\begin{figure}[h]
    \centering
    \includegraphics[width=0.45\textwidth]{../images/plots/results/fig4_drawdown.pdf}
    \caption{Drawdown profiles for all four strategies (2015--2025).}
    \label{fig:drawdown}
\end{figure}

\subsubsection*{Which strategy won overall, and by how much?}
On a raw return basis, Equal
Weight was the clear winner, delivering a total return of 91.57\% (CAGR: 6.08\%) over the period. This result is more than three times the return of the next best strategy, Risk Parity (27.34\%, CAGR: 2.22\%). MVO and Black-Litterman lagged further still at 23.56\% and 22.17\% respectively. The cumulative return chart reinforces this: Equal Weight’s portfolio value surpassed £190k by late 2025, while the others clustered between £135k–£163k, with Risk Parity the weakest performer in absolute terms.

\subsubsection*{Does the ranking change on a risk-adjusted basis?}
The ranking is preserved but compressed. Equal Weight retains the top position with a Sharpe ratio of 0.44 and Sortino of 0.56, reflecting that its higher volatility (13.94\%) was more than compensated by superior returns. Risk Parity comes second (Sharpe: 0.24, Sortino: 0.29), followed by MVO (0.19\,/\,0.23) and Black-Litterman (0.15\,/\,0.18).

Notably, the optimised strategies (MVO and Black-Litterman), which are theoretically intended to maximise risk-adjusted returns, actually underperformed the naive Equal Weight benchmark across all metrics. This is a well-documented phenomenon was mentioned above and is called the \textbf{``1/N puzzle''}. In practice, estimation error in expected returns and covariances often causes optimised portfolios to underperform simpler allocations out-of-sample.

\subsubsection*{What does the drawdown chart reveal that the cumulative returns do not?}
The cumulative returns chart presents a broadly upward narrative, but the drawdown chart tells a more nuanced story about the cost of participating in that growth.


\medskip
\noindent\textbf{\textit{Equal Weight's hidden fragility.}} Despite delivering the highest returns, Equal Weight also experienced the largest maximum drawdown at $-29.68\%$, most visibly during the COVID-19 market crash in early 2020. At that point, a \pounds100,000 portfolio would have temporarily lost nearly \pounds30,000 of value --- a risk that is easily missed in the smooth upward trajectory of the cumulative chart.

\medskip
\noindent\textbf{\textit{Equal Weight's hidden fragility:}} Despite delivering the highest returns, Equal Weight also experienced the largest maximum drawdown at $-29.68\%$, most visibly during the COVID-19 market crash in early 2020. At that point, a \pounds100k portfolio would have temporarily lost nearly \pounds30k of value.

\medskip
\noindent\textbf{\textit{Risk Parity's resilience under stress:}} Risk Parity and MVO share the best max drawdown ($-22.50\%$), and the drawdown chart shows Risk Parity recovered relatively quickly from most dips. Its comparatively low volatility (9.39\%) results in drawdowns that are both milder and shorter in duration, which is valuable for investors with lower loss tolerance or shorter time horizons.

\medskip
\noindent\textbf{\textit{Black-Litterman's 2022--2024 struggle:}} The drawdown chart highlights a sustained decline of over $-22\%$ between 2022 and 2024. This risk episode is partially masked in the cumulative return plot due to the strong performance recorded in 2020--2022.

%% ─────────────────────────────────────────────
\subsection{Risk-Return Trade-off}

The previous section showed that Equal Weight delivered the strongest absolute returns over the period. This section examines whether that outperformance holds up when risk is taken into account, and whether any strategy maintains a consistent edge throughout the sample or simply benefits from favourable conditions at certain points in time. Two figures inform this discussion: a risk-return scatter showing where each strategy sits in volatility-return space over the full period, and a rolling 12-month Sharpe ratio tracking how risk-adjusted performance evolved through time.
 
\begin{figure}[h]
    \centering
    \includegraphics[width=0.45\textwidth]{../images/plots/results/fig5_risk_return_scatter.pdf}
    \caption{Risk-return scatter (one dot per strategy, volatility on x-axis, CAGR on y-axis).}
    \label{fig:risk_return_scatter}
\end{figure}

\begin{figure}[h]
    \centering
    \includegraphics[width=0.45\textwidth]{../images/plots/results/fig6_rolling_sharpe.pdf}
    \caption{Rolling 12-month Sharpe ratio (all four strategies, full sample period).}
    \label{fig:rolling_sharpe}
\end{figure}

\subsubsection*{Positioning in Risk-Return Space}
Equal Weight occupies the upper-right quadrant, offering the highest CAGR ($\sim$6.0\%) but also the highest volatility ($\sim$14.0\%). Risk Parity takes the opposite stance with the lowest volatility ($\sim$10\%) and $\sim$3.4\% CAGR. Black-Litterman sits in the middle at $\sim$4\% CAGR, though at volatility nearly as high as Equal Weight, offering little reward for that additional risk. MVO ends up being the most difficult to justify: higher volatility than Risk Parity yet the weakest return of all four ($\sim$2.7\%), placing it in a vulnerable position in risk-return space.

\subsubsection*{Consistency of Risk-Adjusted Performance Over Time}
The rolling 12-month Sharpe ratio reveals that no strategy maintains a persistent performance advantage across the full sample. All four strategies move largely in tandem, suggesting that the choice of strategy offers limited protection during systemic shocks and that the macroeconomic regime is a stronger determinant of short-term risk-adjusted performance than the allocation model itself.

%% ─────────────────────────────────────────────
\subsection{Regime Dependence}
Full-period metrics can mask very different behaviour across market conditions. This section stress-tests the four strategies across three distinct episodes: the Brexit shock (Jun--Dec 2016), the COVID-19 crash (Feb--Sep 2020), and the 2022 rate-hiking cycle (Jan--Dec 2022), each representing a different type of market disruption. Table 3 and Figure 7 summarise cumulative returns across these periods.

\begin{center}
\begin{table}[h]
\caption{Strategy Performance During Key Market Events}
\label{tab:stress_periods}
\begin{tabular}{llrrrr}
\toprule
\textbf{Event} & \textbf{EW} & \textbf{MVO} & \textbf{BL} & \textbf{RP} \\
\midrule
Brexit     &  15.95\% &   0.84\% &  20.28\% &   6.16\% \\
COVID      & -25.84\% &   3.59\% & -13.44\% & -13.69\% \\
Rate Hikes &   1.50\% &  -6.02\% & -10.90\% &  -4.34\% \\
\bottomrule
\end{tabular}
\end{table}
\end{center}

\begin{figure}[h]
    \centering
    \includegraphics[width=0.45\textwidth]{../images/plots/results/fig7a_regime_brexit.pdf}
    \caption{Regime: Brexit (Jun--Dec 2016).}
    \label{fig:regime_brexit}
\end{figure}

\begin{figure}[h]
    \centering
    \includegraphics[width=0.45\textwidth]{../images/plots/results/fig7b_regime_covid.pdf}
    \caption{Regime: COVID-19 (Feb--Mar 2020).}
    \label{fig:regime_covid}
\end{figure}

\begin{figure}[h]
    \centering
    \includegraphics[width=0.45\textwidth]{../images/plots/results/fig7c_regime_2022.pdf}
    \caption{Regime: Rate Hikes (Jan--Dec 2022).}
    \label{fig:regime_2022}
\end{figure}

\subsubsection*{Brexit (Jun--Dec 2016)}
The Brexit referendum triggered a sharp sterling depreciation and a rapid rotation into exporters and internationally-exposed equities. Black-Litterman was the standout performer, gaining 20.25\% over the period. Its view-adjusted allocations appear to have captured the post-referendum equity rebound effectively. Equal Weight followed at 15.95\%, benefiting from its broad exposure across assets without concentration risk pulling it in the wrong direction. Risk Parity lagged considerably at 6.16\%, its defensive, risk-equalising stance limiting upside participation during a strong directional move. MVO was essentially flat at 0.84\%, suggesting its concentrated positioning was poorly aligned with the assets that benefited most from the shock.

\subsubsection*{COVID-19 (Feb--Sep 2020)}
The COVID crash was sudden and indiscriminate, hitting all strategies hard in March 2020. What made a real difference is how quick and complete their recovery was. MVO recovered fastest and was the only strategy to finish the period in positive territory (+3.59\%), suggesting its concentrated positioning happened to align with assets. Black-Litterman and Risk Parity ended the period in almost identical territory ($-13.44\%$ and $-13.69\%$ respectively), but went under different paths. Black-Litterman clawed its way back to breakeven by June 2020 before a second selloff dragged it back down, while Risk Parity simply never recovered, staying underwater for the entire period without any meaningful rebound. Equal Weight was the worst performer by far, ending down $-25.84\%$. Its broad equity exposure amplified the drawdown, and the chart shows it had not recovered anywhere near its starting level by October 2020.

\subsubsection*{Rate Hikes (Jan--Dec 2022)}
2022 was uniquely damaging: equities and bonds sold off together, breaking a well-established negative correlation between these assets. Equal Weight was the most resilient, ending the year roughly flat at $+1.5\%$. Its lack of explicit bond exposure weighting meant it was less exposed to the bond sell-off than the more sophisticated models. Risk Parity ($-4.34\%$) held up better than its structural weakness would suggest. While the equity-bond correlation breakdown directly undermines its core diversification logic, losses were contained to a level that outperformed both MVO and Black-Litterman. This suggests that the risk spreading across other assets have provided a cushion. MVO ($-6.02\%$) fared worse than Risk Parity despite having no structural sensitivity to bond markets. Black-Litterman was the worst performer at $-10.9\%$, its view-adjusted allocations, effective during Brexit, proving ill-suited to the speed and severity of the 2022 repricing.

%% ─────────────────────────────────────────────
\subsection{Portfolio Characteristics}

This section looks under the hood at how each strategy allocates capital and how much it trades to get there, two practical considerations that matter significantly for real-world implementation.

\begin{figure}[h]
    \centering
    \includegraphics[width=0.45\textwidth]{../images/plots/results/fig8_allocation_heatmaps.pdf}
    \caption{Allocation heatmaps for MVO, Black-Litterman, and Risk Parity (assets on y-axis, time on x-axis) --- Equal Weight omitted as trivially flat.}
    \label{fig:allocation_heatmaps}
\end{figure}

\begin{figure}
    \centering
    \includegraphics[width=0.45\textwidth]{../images/plots/results/fig9_turnover.pdf}
    \caption{Average monthly turnover for all four strategies.}
    \label{fig:turnover}
\end{figure}

\subsubsection*{Allocation Stability}

The heatmaps reveal stark differences in how each strategy distributes and maintains its allocations over time.

MVO is visibly the most unstable. Allocations shift dramatically from period to period, with heavy concentration rotating between a handful of assets. Most notably IGLT.L and SGLD.L dominating at different points, while the majority of the portfolio sits near zero for extended stretches. This ``all or nothing'' behaviour is characteristic of mean-variance optimisation, which tends to produce corner solutions that look very different from one rebalancing period to the next. In practice, this makes the portfolio difficult to manage and exposes investors to sharp, unintended shifts in risk profile between rebalancing dates.

Black-Litterman appears to be smoother, with weights more evenly distributed across assets and less violent rotation between periods. IGLT.L consistently attracts a high allocation from 2018 onwards, which is worth noting as a point of concentration risk since a single asset commanding a persistently large share of the portfolio undermines the diversification the model is designed to provide.

Risk Parity is the most stable of the three, with allocations broadly uniform across assets for most of the period. This is consistent with its design objective of equalising risk contribution rather than capital. The periodic spikes in IGLT.L weight reflect its lower historical volatility, which mechanically results in a higher risk-budget allocation within the framework. For an investor, this predictability is attractive and reduces operational uncertainty.

\subsubsection*{Turnover and the Cost of Complexity}

High turnover is not a sign of a well-functioning strategy, it is a cost. The turnover chart shows MVO trading at an average monthly rate of 34.7\%, by far the highest of the four: more than twice Black-Litterman (12.7\%), nearly four times Risk Parity (9.0\%), and almost fourteen times Equal Weight (2.5\%).

Every rebalancing generates transaction costs such as brokerage fees and bid-ask spreads, which erode net returns. Given that MVO already delivered the weakest risk-adjusted performance on a gross basis, the addition of implementation costs would make its net performance even less competitive. MVO requires the greatest effort and cost to implement, yet provides the lowest return in comparison.

Equal Weight, by contrast, requires minimal intervention, yet outperforms all three more complex strategies over the full period. This is perhaps the most pointed finding of the section: \textbf{complexity does not necessarily pay.}


\section{Discussion}
Summarize the main conclusions of your research. Reiterate the key insights and their importance.


\subsection{Why did 1/N perform well?}


\subsection{The risk parity puzzle}


\subsection{Practical Implications}



\subsection{Limitations}


\section{Conclusion}



\section{Declarations}
Briefly describe the contribution of each author to the research and writing of the report. For example: "A.B. designed the research. C.D. collected the data. E.F. performed the analysis. All authors contributed to writing the report."



Future work could include:
\begin{itemize}
    \item Exploring alternative methodologies or datasets.
    \item Addressing limitations of the current study.
    \item Expanding the research to a different market or asset class.
\end{itemize}


\begin{table}[bt!]
\caption{Descriptive caption for your table.}\label{tab:example_table}
\begin{tabular}{l c c c}
\toprule
Category & Metric 1 & Metric 2 & Metric 3 \\ 
\midrule
Group A & 0.00 & 0.00 & 0.00 \\ 
Group B & 0.00 & 0.00 & 0.00 \\ 
Group C & 0.00 & 0.00 & 0.00 \\ 
\bottomrule
\end{tabular}

\begin{tablenotes}
\item Note: Explain any specific details about the data in the table.
\end{tablenotes}
\end{table}

\bibliography{../refs/refs}

\end{document}

