\documentclass[a4paper,num-refs]{scr-contemporary} % Use our custom class

\usepackage{xcolor}
\usepackage{graphicx}
\usepackage{siunitx}
\usepackage{amsmath} % For advanced mathematical typesetting

% AP Capital Research Specifics
\jname{Surrey Capital Research} % Set the journal name to Surrey Capital Research

\secondlogo{../images/scr-logo.pdf} % Placeholder for University of Surrey logo
\jlogo{../images/surrey-logo.pdf} % Placeholder for SCR logo


\title{Research Report Title}

\author{Author Names}
% \author{Author Name} % Add more authors as needed

\affil{Surrey Capital Research, University of Surrey}
\papercat{Quantitative Research}

\runningauthor{Surrey Capital Research}

\jvolume{01}
\jnumber{01}
\jyear{2025}

\begin{document}

\begin{frontmatter}
\maketitle
\begin{abstract}
This abstract should summarize the key aspects of the report. It should concisely outline the research 
background, methodology, main findings, and conclusions. Aim for approximately 250 words, avoiding citations 
and minimizing abbreviations.

\textbf{Background}: Introduce the topic and the problem your research addresses.
\textbf{Methodology}: Describe the methods and data used in your analysis.
\textbf{Results}: Highlight the main outcomes and observations.
\textbf{Conclusions}: Summarize the key insights and implications of your findings.
\end{abstract}

\begin{keywords}
Keyword1; Keyword2; Keyword3; Keyword4; Keyword5
\end{keywords}
\end{frontmatter}

\begin{keypoints*}
\begin{itemize}
\item Key finding or takeaway 1.
\item Key finding or takeaway 2.
\item Key finding or takeaway 3.
\end{itemize}
\end{keypoints*}

\section{Introduction}
\subsection{The Portfolio Allocation Problem}
\indent \indent The portfolio allocation problem is the fundamental question in investment management. 
It asks: How should capital be distributed across available investment opportunities to best achieve 
an investor's objectives? \\
\indent The problem was intractable until 1952, when Harry Markowitz published \textit{Portfolio Selection} in the 
\textit{Journal of Finance} and turned it into a clean optimisation problem, and the “efficient frontier” 
gave investors a principled answer. Markowitz's mean variance model minimises portfolio variance for a given 
level of expected return. It was revolutionary in the consideration of variance as the risk, standing as a 
hallmark of modern portfolio theory. In practice however, the model has proven difficult to implement reliably. \\
\indent Michaud (1989) challenged this view directly, arguing that small estimation errors in expected 
returns produce wildly unstable portfolio recommendations. \\
\indent In 2009, DeMiguel et al. delivered a provocative result: he found that even sophisticated strategies 
incorporating shrinkage estimators and Bayesian methods failed to consistently outperform 1/N out-of-sample, 
suggesting that estimation error dominates theoretical optimality for typical portfolio problems. This tension 
between the theoretical elegance of optimisation and its empirical fragility motivates the present study. 


\subsection{Background and Research Objectives}
\indent \indent This study addresses three main questions:
\begin{enumerate}
    \item Do optimisation models outperform a naive 1/N benchmark
    out-of-sample, applied to a multi-asset UK portfolio over 
    a ten-year period? 
    \item Which model offers the best risk-adjusted performance, 
    as measured by the Sharpe ratio?
    \item How does model performance vary across distinct market
    regimes: the Brexit referendum (2016), the COVID-19 crash (2020),
    and the 2022 rate-hiking cycle?
\end{enumerate}

\subsection{Summary of Findings}
\subsection{Structure of the Report}
\indent \indent The remainder of this report is organised as follows: \newline
Section 2 reviews the theoretical foundations of each model 
and derives their key mathematical results. Section 3 
describes the data, implementation choices, and backtesting 
framework. Section 4 presents the empirical results, including 
full-period performance metrics and a breakdown by market 
regime. Section 5 discusses the findings and their practical 
implications. Section 6 concludes.

\section{Methodology}

This section details the design of your study, the data used, the models applied, and the analytical framework. 
The goal is to provide enough detail for another researcher to replicate your work.

\subsection{Data Acquisition and Processing}
Describe the data you used, where you got it from (e.g., Bloomberg, Refinitiv, web scraping), and any steps you 
took to clean, process, or transform it. Mention the time period and frequency of the data.

\subsection{Analytical Models}
Describe the quantitative or qualitative models used in your analysis. Include mathematical formulations if 
applicable. For example, if you are using a regression model, you would specify it here:
\begin{equation}
Y = \beta_0 + \beta_1 X_1 + \beta_2 X_2 + \epsilon
\end{equation}
Explain the variables and the assumptions of the model.

\subsection{Backtesting or Validation Framework}
If you are testing a strategy, describe the backtesting framework. Mention key parameters like rebalancing 
frequency, transaction cost assumptions, and performance metrics used.

\section{Results and Discussion}

This section presents the key findings from your analyses and interprets their significance \cite{example2023}. Use figures and tables to present your results clearly.

\subsection{Primary Findings}
Present your main results. This could be in the form of tables summarizing statistical outputs, or charts showing trends and relationships.

\begin{figure}[bt!]
\centering
% \includegraphics[width=\linewidth]{path/to/your/figure.pdf} % Example: \includegraphics[width=0.8\linewidth]{figures/performance_chart.pdf}
\caption{Descriptive caption for your figure.}\label{fig:example_figure}
\end{figure}

\begin{table}[bt!]
\caption{Descriptive caption for your table.}\label{tab:example_table}
\begin{tabular}{l c c c}
\toprule
Category & Metric 1 & Metric 2 & Metric 3 \\ 
\midrule
Group A & 0.00 & 0.00 & 0.00 \\ 
Group B & 0.00 & 0.00 & 0.00 \\ 
Group C & 0.00 & 0.00 & 0.00 \\ 
\bottomrule
\end{tabular}
\begin{tablenotes}
\item Note: Explain any specific details about the data in the table.
\end{tablenotes}
\end{table}

\subsection{Interpretation of Findings}
Discuss what your results mean. How do they relate to your initial research question? Are they consistent with existing literature? What are the implications of your findings?

\section{Conclusion and Future Work}

Summarize the main conclusions of your research. Reiterate the key insights and their importance.

Future work could include:
\begin{itemize}
    \item Exploring alternative methodologies or datasets.
    \item Addressing limitations of the current study.
    \item Expanding the research to a different market or asset class.
\end{itemize}

\section{Declarations}

\subsection{Author Contributions}
Briefly describe the contribution of each author to the research and writing of the report. For example: "A.B. designed the research. C.D. collected the data. E.F. performed the analysis. All authors contributed to writing the report."

\subsection{Competing Interests}
The authors declare that they have no competing interests.

\subsection{Funding}
This research was conducted as part of the AP Capital Research initiative at the University of Surrey and received no external funding.

\subsection{Acknowledgements}
Thank anyone who provided help or support for the project, such as professors, mentors, or other colleagues.

%% Specify your .bib file name here, without the extension
\bibliography{../refs/apcr-refs}

\end{document}
